\section{Introduction}
   Modern computing needs have grown to require the execution of larger and more complex tasks, which require more storage and computational power than ever before.  Cloud computing has emerged as both a topic in research and industry, and strives to provide the best platform for these tasks given a set of resources to run them on.  Cloud computing is closely tied with its roots in distributed and grid computing systems and has significant overlap with them.  

  Eucalyptus is a popular cloud computing system, which allows for a transparent deployment of virtual machines (VMs) across a large variety of systems (usually in the form of a cluster or set of clusters) and also provides a consistent interface to access these systems.  From the perspective of Eucalyptus, the contents of an individual VM is seen only as a black box.  Despite this limited knowledge, Eucalyptus is tasked with scheduling these VMs with some level of optimality.  Finding a good configuration of these VMs is critical to achieving high utilization of the systems resources or consuming less power.  Eucalyptus itself has no control over when new VMs are requested and it also has no way to predict when they will be shut down (and removed from the system).  Eucalyptus uses a scheduler to assign VMs to specific computers, which has three different modes: a greedy scheduler, which places the VM in the first available computer it can find, a round-robin scheduler, which attempts to balance the load across machines by distributing the VMs between the computers in a cluster, and a power-saving mode, which uses the greedy scheduler for allocation, but also looks at usage thresholds and attempts to power down machines.

  Although Eucalyptus goes a long way in the way of providing transparent deployment and management.  Unfortunately, its static scheduler is very limited in its ability.  VMs can run for days, weeks, months, or years.  Additionally, these VMs can stress or saturate none or all of the resources provided to them.  Eucalyptus (or any other VM management system) has only a limited view of the VMs, and cannot be expected to know details of their execution or duration.  As new VMs are added and removed or their execution patterns change, the configuration in which Eucalyptus placed the VMs will quickly diverge from an optimal configuration.

For example, if there are 4 VMs to be scheduled across two machines, Eucalyptus will distribute them across these machines in a reasonable fashion.  With the round-robin scheduler, it would likely end up with 2 VMs on each of the two machines.  After this, it has no ability to change the system.  Even without the addition of new VMs or the removal of existing VMs, there are many ways in which this is non-optimal.  If the 2 VMs on the first machine has little to no CPU load and the 2 VMs on the second machine are saturating the CPU, it is obvious that this configuration could be improved.  This is not limited to the CPU, as the disk, memory, network links, and other computing hardware are also shared between the VMs.

VM migration allows VMs to be moved between different computers, and live migration is a subset of VM migration, where the migration is transparent to an outside observer.  As Eucalyptus is unaware of the execution, connections, etc associated with the VMs, it is not able to assume a non-live migration is acceptable.

%mostly stolen from the abstract, might want to diverge more
We present \emph{Koala}, an enhanced Eucalyptus system which uses live migration and a dynamic scheduler to greatly improve resource utilization within a cluster.  Utilizing the benefits provided from live migration, Koala's dynamic scheduler can make effective and reactive scheduling decisions that were not possible by the basic Eucalyptus system.  Additionally, Koala's use of learning algorithms allow it to optimize the configuration of the scheduler and make it robust to changing needs and demands on the cluster.


  

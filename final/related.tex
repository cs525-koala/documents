\section{Related Work}

%% Eucalyptus scheduler, talk about it here
The foundation of our work is the set of schedulers contained in
Eucalyptus~\cite{Eucalyptus}.  Eucalyptus is a cloud computing infrastructure,
which uses virtualization technologies such as Xen~\cite{Xen},
VMware~\cite{VMware} and KVM/QEMU~\cite{QEMU}, and provides an interface to
EC2/S3 ~\cite{AWS}.   Eucalyptus ships with two main scheduling algorithms: a
greedy algorithm and a round-robin algorithm.  The greedy algorithm will use the
first available node that can run a given job, and round-robin will find the
\emph{next} suitable node.  Eucalyptus also implements a power-saving scheduling
algorithm based on the greedy scheduler that attempts to shut down idle nodes in
order to conserve power.  Note that these schedulers are only used to determine
where to put a start a job, because Euclayptus does not support live migration.
Koala extends Eucalyptus to support live migration, enabling much more
aggressive and reactive scheduling algorithms.  Additionally Koala uses dynamic
information about the system as well as a significantly more sophisticated
parameterized scheduling algorithm resulting in a more desirable use of
resources.

%% Papers we read for class.
Researchers from Microsoft built Quincy~\cite{Quincy}, a cloud scheduling
framework for Dryad and MapReduce-like systems.  They reduced the scheduling to
a min-cost problem in order to expose a variety of adjustable options.  They
found these options were effective in tuning the scheduler to perform various
workloads.  In our project, we will build a similarly parameterized scheduler,
but will target Eucalyptus, which has fundamentally different scheduling
requirements.  Additionally our work focuses on building a tool that determines
the optimal parameters, not just the scheduling framework itself.

Zaharia et al. developed their LATE scheduler for MapReduce in order to improve
task speculation on heterogenous systems~\cite{Zaharia}.  This contrasts our
work because we target not just the service level (MapReduce) but the
infrastructure itself (at the VM level), where speculation is not traditionally
used.  Similarly Mantri et al.~\cite{Mantri} focus on improving performance by
better handling of stragglers  in a MapReduce system.  Mantri et al. focus on
exploring the cause of an outlier through monitoring and other methods to take
appropriate action and free those resources.  Early detection of outliers is key
to their methodology and is effective for MapReduce workloads.

%% Zhao work on adaptive load-balancing.
Load-balancing is a critical issue related to the optimized scheduling.  Zhao,
et. al ~\cite{Zhao} developed an adaptive load-balancing system algorithm for
Eucalyptus called ``COMPARE\_AND\_BALANCE''.  Their system focuses on
load-balancing as the primary goal, while Koala provides a more general
scheduler to fit the needs and success metrics unique to each cluster.
Additionally they focused on a distributed algorithm forbidding the use of a
centralized scheduling node, which is not a limitation we impose in our system.

%% MLN + Internal VM polity + Migration

Begnum et al.~\cite{Begnum} have experimented with moving virtual machine policy
inside the virtual machine itself.  The benefits of this include improved
portability and aptitude for live-migration.  Additionally, this increases the
ease with which virtual machines can move between non-homogeneous systems
in a cloud.  They worked with MLN (Manage Large Networks), an open source tool
providing atomic operations to enable the management of large quantities of a
variety of virtual machines \cite{Xen,UML,VMware}.  It provides an interface in
the form of a configuration language.

% GA-based approaches:

%% Topology-Aware Resource Allocation for Data-Intensive Workloads
A number of researchers have applied genetic algorithms (GA) to the task of
dynamically scheduling on the cloud~\cite{Lee,Zhong,Chenhong}.  In~\cite{Lee}
they improved upon Eucalyptus's scheduling algorithms with a focus on topology
sensitivity using GA.  They claim in their future work they would be interested
in machine learning based systems, similar to our system.  Additionally they
focused on the data-intensive workload produced from MapReduce tasks. Zhong, et
al.~\cite{Zhong} built their Improved Genetic Algorithm (IGA) to improve the
scheduling algorithms commonly used in open-source clouds.

%% Scientific workflows:

Some scheduling algorithms focus on particular workloads.  While some
researchers only consider MapReduce-like loads, others focus on scientific
workloads \cite{Juve,Simmhan,Hoffa}.  By gathering basic characteristics of the
workload, Simmhan et al.~\cite{Simmhan} build a \emph{blackbox} system to help
schedule the workflow.  This is similar to our work, although we hope to
determine the characteristics of the workflow experimentally, and not require
them as inputs from the user, and instead only have the user indicate their
desired success metrics.

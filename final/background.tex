\section{Background}
\label{sec:background}

Koala runs on Eucalyptus, a system which provides an Infrastructure as a
Service (IaaS) cloud interface.  This exposes virtualized hardware resources to
the user, allowing them to build and manage the system as they see fit.
Eucalyptus is made of a hierarchy of controllers, including the Node
Controllers (NCs), Cluster Controllers (CCs), Cloud Level Controller (CLC),
Storage Controllers (SCs), and the Walrus Storage Controller.  They communicate
through web front ends, generated by Web Services Description Language (WSDL).
An overview of these components and how they interacted is shown in
Figure~\ref{fig:Eucalyptus}.

\scalefig{Eucalyptus}{0.9}{Eucalyptus System Overview}

\subsection{Node Controller}
Node Controllers are at the bottom of the Eucalyptus hierarchy.  They are responsible for communicating with the hypervisor running on the node through the libvirt virtualization library.  Prior to our modifications and ignoring inter-VM communication, the Node Controllers communicate exclusively with their respective Cluster Controller.  They maintain the state of the VM and the host system's resources and communicate them back to the Cluster Controller.

\subsection{Cluster Controller}
Cluster Controllers are responsible for maintaining the state of a number of nodes.  The CC sends heartbeats and queries information about the instances and resources on each node and passes some of this information up to the CLC.  The CC is responsible for scheduling VMs to the nodes it maintains.  In order to reduce network overhead as Eucalyptus scales, the CC caches information about the nodes and VMs so it does not have to poll the nodes as frequently.

\subsection{Cloud Level Controller}
The Cloud Level Controller is responsible for managing the various controllers in the Cloud and providing a web front-end to administer the Cloud.  The CLC talks with the CCs, SCs, and Walrus Controller and communicates and interfaces between them.  It also manages network resources and is responsible for many of the security functionality of Eucalyptus.

\subsection{Storage Controller}
A Storage Controller would normally be group with each cluster in a cloud.  The SC exposes a block level storage, which the VMs can mount and store persistent data.  The SC is designed to implement the Amazon Elastic Block Storage (EBS) specification.

\subsection{Walrus Controller}
Walrus is a Eucalyptus's put-get bucket storage system.  It is designed to be compatible with Amazon's S3 storage system.  Eucalyptus uses Walrus for storage of user data and for access control of VMs.

\subsection{Static Scheduling}
%% Eucalyptus scheduler, talk about it here
The foundation of our work is the set of schedulers contained in
Eucalyptus~\cite{Eucalyptus}.  Eucalyptus is a cloud computing infrastructure,
which uses virtualization technologies such as Xen~\cite{Xen},
VMware~\cite{VMware} and KVM/QEMU~\cite{QEMU}, and provides an interface to
EC2/S3 ~\cite{AWS}.   Eucalyptus ships with two main scheduling algorithms: a
greedy algorithm and a round-robin algorithm.  The greedy algorithm will use the
first available node that can run a given job, and round-robin will find the
\emph{next} suitable node.  Eucalyptus also implements a power-saving scheduling
algorithm based on the greedy scheduler that attempts to shut down idle nodes in
order to conserve power.  Note that these schedulers are only used to determine
where to put a start a job, because Euclayptus does not support live migration.
Koala extends Eucalyptus to support live migration, enabling much more
aggressive and reactive scheduling algorithms.  Additionally Koala uses dynamic
information about the system as well as a significantly more sophisticated
parameterized scheduling algorithm resulting in a more desirable use of
resources.


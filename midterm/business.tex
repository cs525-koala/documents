\section{Business Plan}
  Modern computing is moving towards larger and more complex tasks.  These tasks require more storage and computational power than ever before.  Cloud Computing is a developing field, built around providing solutions to this problem.  Cloud Computing is closely tied with distributed and grid computing, and shares technology and infrastructure with them.  Many advancements have been made in these areas, and Cloud Computing is currently a very popular venture for computing start-ups.

  Eucalyptus is an infrastructure provided by Eucalyptus Systems Inc, which allows for deployment across a large variety of systems and provides a consistent interface for access to the underlying network.  The current version of Eucalyptus has a primitive scheduler, utilizing greedy and round robin algorithms.  In order to fully utilize a large cluster of machines, properly delegating tasks and balancing resource utilization is key.

  We propose Koala, an enhanced Eucalyptus system, which uses a learning scheduler to maximize throughput and utilization of the machines in a cluster.  Koala's algorithm will gather scheduling events during operation and use this data to adjust how the scheduler behaves.  Our system would allow for increased utilization of machines in a cluster, reducing the hardware, energy, and management expenses associated with the operation of a large cluster of machines.

  In addition to the automated improvements to efficiency and throughput, the Koala scheduler can be influenced externally to account for additional factors.  It can be instructed to minimize network bandwidth (for other tasks) or reduce power consumption to reduce operation costs.  For example, if resources are not scarce, or a workload is not time critical, the scheduler can be instructed to prioritize efficiency, and not use a machine if it cannot sufficiently utilize that machine's resources.
 
  The current version of Eucalyptus is currently only capable of static scheduling.  Due to churn, these allocations can drift away from optimal configurations.  For example, if a 100 instances were running on 100 machines at 25% utilization and energy efficiency was important, the Koala scheduler could migrate the instances to just 25\% of the machines, and the remaining 75\% of the machines could be powered off in order to conserve energy.  In the case were the optimal throughput was desired, it could perform the same operation in reverse.  

  Additionally, during operation, the needs and priorities that the original scheduling was based on can change.  A company would generally want to conserve power (and as a result, money), wherever possible.  However, a deadline may approach, which would be unreachable at a current schedule.  The ability to reconfigure this dynamically allows the cluster to be optimized for whatever the current need is, independent of whatever the original need was.

  A dynamically scheduled system offers so many benefits over a statically scheduled system.  It is not unreasonable to assume that any company who invests enormous amounts of capital in a cluster intends to use it, and doing so efficiently can enormous amounts of money in reduced costs of hardware, energy and management.  Koala provides these companies the opportunity to do exactly this.
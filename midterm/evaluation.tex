\section{Evaluation}

Our current setup for Koala uses three computers.  The Eucalyptus Cloud Level Controller, Cluster Controller, Walrus Controller, and Storage Controller all are running on a CentOS 5 machine, with a 3GHz dual core processor and 8GB of ram.  We have two laptops running the Node Controllers on a Debian Squeeze machine, one with a 2.4GHz dual core processor and 4GB of ram, and the other with a 1.8GHz dual core processor and 2GB of ram.  The node controllers both run instances in QEMU with KVM support.  

In addition to changing the scheduling frequency, our current scheduler can be configured to either optimized for throughput or energy efficiency.  We conducted the following two experiments by varying the threshold between throughput and energy efficiency and using a scheduling frequency of 30 seconds.

In our first test, we configure our scheduler to maximize throughput.  We start three VMs, of which two are placed on one node and one on the other.  We then terminate the VM (asynchronously through the web interface) on the node with only one, so one node has two VMs and the other node has none.  At the next iteration of or scheduler, one of the two VMs on other node is migrated to the empty node.  When configured for throughput, the scheduler attempts to run the fewest VMs per core per machine.  The scheduler operates attempting to balance the ratio of cores utilized across all machines, and is robust both machines with varying numbers of cores and VMs configured to use varying numbers of cores.

In our second test, we configure our scheduler to optimize energy consumption.  We again start with three VMs, of which two are placed on one node and one on the other.  Now, we terminate one of the VMs (again asynchronously through the web interface) on the node with two, so each of the two nodes is left with a single instance.  At the next iteration of the scheduler, one of the two nodes migrates its VM to the other node, leaving it without any VMs, and it could then be suspended or powered off.  We again expect this to be robust to larger and homogeneous systems where the VMs and nodes both support varying numbers of cores.

Finally, to test the robustness of the scheduling components and particularly the live migration feature we added to Eucalyptus, we ran an experiment with a special scheduler designed to randomly move VM's around amongst the available resources.  We ran this experiment for over an hour an a half, resulting in over 180 migrations.  At the end of the experiment, all of the migrations were successful.  The guest VM's were still running, and the node controllers and cluster controllers all had clean state, with no failures in any of the migration attempts.  There was no failure at an hour and a half, but we believed that long enough to demonstrate the robustness and completeness of our live migration support implementation.

Future evaluation will be performed to verify these beliefs, as we get access to additional machines.  As the scheduler and learning algorithms are more fully implemented, we will begin to compare our implementation to a default Eucalyptus system.


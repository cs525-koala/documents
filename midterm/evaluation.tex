\section{Evaluation}

Our current setup for Koala uses three computers.  The Eucalyptus Cloud Level Controller, Cluster Controller, Walrus Controller, and Storage Controller all are running on a CentOS 5 machine, with a 3GHz dual core processor and 4GB of ram.  We have two laptops running the Node Controllers on a Debian Squeeze machine, one with a 2.4GHz dual core processor and 4GB of ram, and the other with a 1.8GHz dual core processor and 2GB of ram.  The node controllers both run instances in QEMU with KVM support.  

In addition to changing the scheduling frequency, our current scheduler can be configured to either optimized for throughput or energy efficiency.  

In our first test, we configure our scheduler to maximize throughput.  We start three VMs, of which two are placed on one node and one on the other.  We then terminate the VM on the node with only one, so one node has two VMs and the other node has none.  At the next iteration of or scheduler, one of the two VMs on other node is migrated to the empty node.  In a system with many more VMs and nodes, the scheduler will attempt to run the fewest VMs per core per machine.  The scheduler operates attempting to balance the ratio of cores utilized across all machines, and should be robust both machines with varying numbers of cores and VMs configured to use varying numbers of cores.

In our second test, we configure our scheduler to optimize energy consumption.  We again start with three VMs, of which two are placed on one node and one on the other.  Now, we terminate one of the VMs on the node with two, so each of the two nodes is left with a single instance.  At the next iteration of the scheduler, one of the two nodes will migrate its VM to the other node, leaving it without any VMs, and it could then be suspended or powered off.  We again expect this to be robust to larger and homogeneous systems where the VMs and nodes both support varying numbers of cores.
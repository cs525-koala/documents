\section{Design}

Koala builds upon Eucalyptus in three main ways.  First, Koala extends
Eucalyptus by adding live migration, which serves as a fundamnetal building
block for the other components.  Second, Koala adds a parameterizable scheduler
to the cluster controller that leverages live migration to move tasks around to
achieve some more desirable configuration.  Third, Koala monitors the
performance of the system and uses learning algorithms to tune the parameters to
the scheduler, using this feedback to improve Koala's scheduling decisions.  We
describe each of these components below.

\subsection{Live Migration}
Live migration is the process of transparently moving a running VM between two machines.  In Koala, in order to efficiently manage the resources of a cloud, we need to be able to move VMs around in order to put them in a desired configuration.  Without live migration, this was simply not possible without disrupting the state and/or connections to the machine.  Because of this, implementing live migration was a necessary step in the development of Koala.

\subsection{Parameterized Scheduler}

\subsection{Dynamic Learning}



% For our project, we will modify the scheduler for eucalyptus to have adjustable
% inputs.  Options such as prioritizing of VMs, number of VMs per server,
% resources per VM, migration thresholds and many other options will be
% modify-able.  We will then execute benchmarks and log the events related to the
% scheduler.  We will then run a variety of benchmarks, likely including
% cpu-limited, memory-limited, disk-limited, and network limited benchmarks.  We
% will then analyze these event logs by utilizing data mining techniques.  We will
% use this analysis to generate the scheduler.  We will compare the performance of
% this scheduler to the unmodified scheduler of Eucalyptus.  Given sufficient
% time, we will generate a system that logs evens during execution, which will
% adjust the inputs on the fly, while attempting to maximize performance.

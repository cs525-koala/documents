\section{Related}

%% HOW TO ORGANIZE?! We are writing ~2 pages of related work, and are going to
% have at _least_ 15 references... We need to find a way to group them and
% present them in a somewhat connected manner.

% For now, lets just talk about each paper and try to tie them together in the
% next iteration.

%% Eucalyptus scheduler, talk about it here
The foundation of our work is the set of schedulers contained in
Eucalyptus~\cite{Eucalyptus}.  Eucalyptus is a cloud computing infrastructure,
which uses virtualization technologies such as Xen~\cite{Xen},
VMware~\cite{VMware} and KVM/QEMU~\cite{QEMU}, and provides an interface to
EC2/S3 ~\cite{EC2,S3}.   Eucalyptus ships with two main scheduling algorithms: a
greedy algorithm and a round-robin algorithm.  The greedy algorithm will use the
first available node that can run a given job, and round-robin each time will
find the \emph{next} suitable node.
Eucalyptus also implements a power-saving scheduling algorithm based on the greedy scheduler that
attempts to shut down idle nodes  in order to conserve power.  We believe that
Eucalyptus could benefit with a more specialized scheduler, which takes
additional infomation into account when scheduling virtual machines.
Additionally Eucalyptus does not allow live migration of allocated VMs, which
prevents the exporation of additional scheduling enhancements.  We hope to
address this as well in our work.

%% Papers we read for class.
Researchers from Microsoft built Quincy~\cite{Quincy}, which is a cloud
scheduling framework for Dryad and MapReduce-like systems.  By reducing the
scheduling problem to a customizable min-cost flow problem they expose a number
of knobs that they found were rather effective in making an effective scheduler
for various workloads.  In our project, we will build a similarly
parameterizable scheduler, but targeting Eucalyptus which has fundamentally
different requirements.  Additionally our work focuses on building a tool that
helps you find the optimal parameters, not just the scheduling framework itself.

Zaharia et al. developed their LATE scheduler for MapReduce to improve task
speculation on heterogenous systems~\cite{Zaharia}.  This contrasts our work
because we target not just the service level (MapReduce) but the infrastructure
itself (at the VM level), where speculation is not traditionally used.
Similarly Mantri et al.~\cite{Mantri} focus on the outliers, the stragglers, in
a MapReduce system to improve performance.  Mantri et al. focus on exploring the
cause of an outlier through monitoring and other methods to take appropriate
action and free those resources.  Early detection of outliers is key and is
effective for MapReduce workloads.

%% Zhao work on adaptive load-balancing.
A problem related to optimizing scheduling is that of load-balanacing.  Zhao,
et. al ~\cite{Zhao} developed an adaptive load-balancing system algorithm for
Eucalyptus called ``COMPARE\_AND\_BALANCE''.  This system focuses on
load-balancing as the primary goal, while we hope to build a more general
scheduler to fit the needs and success metrics unique to each cluster, which is
a fundementally different problem.  Additionally they focused on a distributed
algorithm forbidding the use of a centralized scheduling node, which is not a
limitation we impose in our system.

%% MLN + Internal VM polity + Migration

Begnum et all \cite{Begnum} have experimented with moving virtual machine policy
inside the virtual machine itself.  The benefits of this include improved
portability and aptitude for live-migration.  Additionally, this increases the
ease at which virtual machines can move between non-homogeneous systems within a
cloud.  They worked with MLN (Manage Large Networks), an open source tool
providing atomic operations to enable the management of large quantities of a
variety of virtual machines \cite{Xen,UML,VMware}.  It provides an interface in
the form of a configuration language.

% GA-based approaches:

%% Topology-Aware Resource Allocation for Data-Intensive Workloads
A number of researchers have applied genetic algorithms (GA) to the task of
dynamically scheduling on the cloud~\cite{Lee,Zhong,Chenhong}.  In~\cite{Lee}
they improved upon Eucalyptus's scheduling algorithms with a focus on topology
sensitivity using GA.  They claim in their future work they would be interested
in machine learning based systems, similar to our system.  Additionally they
focused on the data-intensive workload produced from MapReduce tasks. Zhong, et
al.~\cite{Zhong} built their Improved Genetic Algorithm (IGA) to improve the
scheduling algorithms commonly used in open-source clouds.


%% Scientific workflows:

Some scheduling algorithms focus on particular workloads.  While some
researchers only consider MapReduce-like loads, others focus on scientific
workloads \cite{Juve,Simmhan,Hoffa}.  By gathering basic characteristics of the
workload, Simmhan et al.~\cite{Simmhan} build a \emph{blackbox} system to help
schedule the workflow.  This is similar to our work, although we hope to
determine the characteristics of the workflow experimentally, and not require
them as inputs from the user.

\section{Related}

%% HOW TO ORGANIZE?! We are writing ~2 pages of related work, and are going to
% have at _least_ 15 references... We need to find a way to group them and
% present them in a somewhat connected manner.

% For now, lets just talk about each paper and try to tie them together in the
% next iteration.

%% Eucalyptus scheduler, talk about it here
The baseline of our work is the default set of schedulers contained in
Eucalyptus~\cite{Eucalyptus}.  Eucalyptus ships with two main schedulers: greedy
and round-robin.  Greedy will simply use the first available node that can run a
given job, and round-robin each time will find the \emph{next} suitable node.
They also implement a power-save scheduler based on the greedy scheduler that
attempts to shut down idle nodes to help conserve power.  These are all basic
scheduling algorithms that are simple and generic.  However we believe that
especially with respect to a given workload much more effective schedulers can
be useful and intend to explore that problem space.  Additionally Eucalyptus's
sechduling framework does not allow the live migration of VM's once they're
allocated, preventing a more complicated scheduling algorithms.  We hope to
address this as well in our work.

%% Papers we read for class.
Researchers from Microsoft built Quincy~\cite{Quincy}, which is a cloud
scheduling framework for Dryad and MapReduce-like systems.  By reducing the
scheduling problem to a customizable min-cost flow problem they expose a number
of knobs that they found were rather effective in making an effective scheduler
for various workloads.  In our project, we will build a similarly
parameterizable scheduler, but targeting Eucalyptus which has fundamentally
different requirements.  Additionally our work focuses on building a tool that
helps you find the optimal parameters.

Systems like those in~\cite{Zaharia} and ~\cite{Mantri} are interesting\ldots

%% Zhao work on adaptive load-balancing.
A problem related to optimizing scheduling is that of load-balanacing.  Zhao,
et. al ~\cite{Zhao} developed an adaptive load-balancing system algorithm for
Eucalyptus called ``COMPARE\_AND\_BALANCE''.  This system focuses on
load-balancing as the primary goal, while we hope to build a more general
scheduler to fit the needs and success metrics unique to each cluster, which is
a fundementally different problem.  Additionally they focused on a distributed
algorithm forbidding the use of a centralized scheduling node, which is not a
limitation we impose in our system.

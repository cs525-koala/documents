\section{Related}

%% HOW TO ORGANIZE?! We are writing ~2 pages of related work, and are going to
% have at _least_ 15 references... We need to find a way to group them and
% present them in a somewhat connected manner.

% For now, lets just talk about each paper and try to tie them together in the
% next iteration.

%% Eucalyptus scheduler, talk about it here
The baseline of our work is the default set of schedulers contained in
Eucalyptus~\cite{Eucalyptus}.  Eucalyptus ships with two main schedulers: greedy
and round-robin.  Greedy will simply use the first available node that can run a
given job, and round-robin each time will find the \emph{next} suitable node.
They also implement a power-save scheduler based on the greedy scheduler that
attempts to shut down idle nodes to help conserve power.  These are all
basic scheduling algorithms that are simple and generic.  However we believe
that especially with respect to a given workload much more effective schedulers
can be useful and intend to explore that problem space.

%% Papers we read for class.
Researchers from Microsoft built Quincy~\cite{Quincy}, which is a cloud
scheduling framework.  Quincy maps the task of job scheduling to a min-cost flow
problem in which they use various cost functions to represent data locality,
cost of killing a task, and more.

``Improving MapReduce performance in heterogeneous systems'' Zaharia~\cite{Zaharia}.

Mantri~\cite{Mantri}.

%% Zhao work on adaptive load-balancing.
A problem related to optimizing scheduling is that of load-balanacing.  Zhao, et
. all~\cite{Zhao} developed an adaptive load-balancing system algorithm
``COMPARE\_AND\_BALANCE''.

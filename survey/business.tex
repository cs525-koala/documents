\section{Business Plan}
  Modern computing is moving towards larger and more complex tasks.  These tasks
  require more storage and computational power than ever before.  Cloud
  Computing is a developing field, built around providing solutions to this
  problem.  Cloud Computing is closely tied with distributed and grid computing,
  and shares technology and infrastructure with them.  Many advancements have
  been made in these areas, and Cloud Computing is currently a very popular
  venture for computing start-ups.

  Eucalyptus is an infrastructure provided by Eucalyptus Systems Inc, which
  allows for deployment across a large variety of systems and provides a
  consistent interface for access to the underlying network.  The current
  version of Eucalyptus has a primitive scheduler, utilizing greedy and round
  robin algorithms.  In order to fully utilize a large cluster of machines,
  properly delegating tasks and balancing resource utilization is key.

  We propose an enhanced system, built on top of Eucalyptus, which uses a
  learning scheduler to maximize throughput and utilization of the machines in a
  cluster.  Our algorithm will gather scheduling events during operation and use
  this data to adjust how the scheduler behaves, based of various metrics such
  as disk and memory utilization, network bandwidth, and resource contention.
  Our software would allow for increased utilization of machines in a cluster,
  reducing the hardware, energy, and management expenses associated with the
  operation of a large cluster of machines.

  The primary innovation behind the design of our scheduler is the utilization
  of machine learning techniques to leverage the event logs generated while
  executing workloads.  For example, if a cluster is network limited, the
  scheduler can assign higher priority to data locality for jobs which have high
  network utilization.  The scheduler will be designed to adjust allocation
  priorities based on a variety of factors, such as the properties of the
  workloads or the current utilization of the cluster.
